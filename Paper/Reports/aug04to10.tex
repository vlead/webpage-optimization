\documentclass[a4paper,10pt]{article}
\usepackage[utf8]{inputenc}

% Title Page
\title{Weekly Report from 4th  to 10th}
\author{Jatin Agarwal}


\begin{document}
\maketitle

\begin{abstract}
Last week we(Nurendra and me) developed a framework to get {\it yslow} reports, hosted at {\it http://54.201.77.123/framework}.
But ylsow report's only list down issues related to client-side performance. But performance depends upon  web page content
(client side), networking and server-side parameters. So we explored {\it webpagetest} tool which take end into accounts
almost all standard metrics and parameters while providing performance. {\it webpagetest} is an open source tool sponsored by Google for analyzing
and finding existing trends in world wide web. So we were exploring {\it webpagetest} from past one week. We did install it on our 
local machine to start with it and later on the container. We used {\it npm} package to install in out machine. We read
and understood it from following website https://www.npmjs.org/package/webpagetest. Later we explored multiple options it
provided.
\end{abstract}

\section{Introduction}
WebPagetest is an open source project that is primarily being developed and supported by Google as part of our efforts to make the web faster.
WebPagetest is a tool that was originally developed by AOL for use internally and was open-sourced in 2008 under a BSD license.
The platform is under active development on GitHub and is also packaged up periodically and available for download if you would
like to run your own instance. The online version at www.webpagetest.org is run by the WPO Foundation for the benefit of the 
performance community with several companies and individuals providing the testing infrastructure around the globe.
We can run a free website speed test from multiple locations around the globe using real browsers (IE and Chrome) and at real consumer
connection speeds. You can run simple tests or perform advanced testing including multi-step transactions, video capture, 
content blocking and much more. Your results will provide rich diagnostic information including resource loading waterfall charts,
Page Speed optimization checks and suggestions for improvements.

The HTTP Archive crawls the world’s top 300K URLs twice each month and records detailed information like the number of HTTP requests,
the most popular image formats, and the use of gzip compression. We also crawl the top 5K URLs on real iPhones as part of the HTTP
Archive Mobile. In addition to aggregate stats, the HTTP Archive has the same set of data for individual websites plus images and 
video of the site loading. Project was started in  2010 and got merged it into the Internet Archive in 2011. The data is collected
using WebPagetest. The code and data are open source. The hardware, mobile devices, storage, and bandwidth are funded by our generous
sponsors:  Google, Mozilla, New Relic, O’Reilly Media, Etsy, Radware, dynaTrace Software, Torbit, Instart Logic, and Catchpoint Systems.
\section{Exploration}
We explored most of the options provided by {\it webpagetest} and found them to be very useful for understanding whole
performance issues with respect to a given url. It provides tones of options that suffices all our performance requirements
for virtual labs. Webpagetest can be used in following tow ways.One option get API key from the admin of {\it webpagetest.org}.
The API key allows around 200 requests per day 
on infrastructure provided by {\it webpagetest.org} and we have used this option for time being.
The major advantage of using infrastructure provided by {\it webpagetest} is that we can do performance analysis from multiple locations
and on all available browsers including internet explorer. 
Other we way is to host private instance of {\it webapgetest} on our infrastructure which will expensive and complex.
We propose this framework to be built using API's of {\it webpagetest}
for measuring performance of virtual labs. We are developing set of shell scripts for building such an automated framework.
Before we proceed we want to
give a presentation on performance and {\it webpagetest} to get a roadmap for performance. So based suggestion and improvements
provided to us we would like to proceed further in developing an automated framework for measuring performance of virtual labs.
Major advantage of using {\it webpagetest} is 

\section{Conclusion}
We propose that {\it webpagetest} should be used for analyzing performance based on requirements at virtual labs. Once we get
performance results from {\it webpagetest} on existing status of all labs then we can start looking for solutions to improve performance.
We can setup an continuous integration system like Travis and Jenkins on our infrastructure that does performance analysis of the labs
periodically to lab developers. 
\end{document}          
